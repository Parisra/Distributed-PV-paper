
% \newgeometry{
% top=25mm,
% }
%\begin{bibunit}

\clearpage
\addcontentsline{toc}{section}{Supplemental Information}
\section*{Supplementary Information}


\subsection*{S1. Pypsa model and power flow optimization}

A brief explanation on the PyPSA model is provided here.
 Comprehensive information about the optimization method, the objective function, and the constraints implemented in the model can be found in previous studies \citeS{neumann_2022assessments, brown2017, horsch2018pypsa} as well as the model documentation \citeS{pypsadocs}. The main goal of the optimization is to minimise the total annualized system costs. These costs include investment costs and operation and maintenance costs for all system components, as shown in Eq.(1), where \(c_{*}\) is capital cost of the component, \(o_{*}\) is operating cost of the component, \(G_{i,r}\) is generator capacity of technology \(r\) at location \(i\), \(E_{i,s}\) is energy capacity of storage \(s\) at location \(i\), \(P_{l}\) is transmission line capacity for line \(l\), \(F_{k}\) is power capacity of technology \(k\) for conversion and transportation of energy, \(g_{i,r,t}\) is generator dispatch of technology \(r\)  at time \(t\), and \(f_{k,t}\) is dispatch of technology \(k\) at time \(t\). Each time snapshot \(t\) is weighted by the time-step \(w_{t}\), and the sum of time-steps is one year.
\begin{equation*}
\min_{G,F,E,P,g,f} = \left[ \sum_{i,r}^{}c_{i,r}.G_{i,r} + \sum_{k}^{}c_{k}.F_{k}+ \sum_{i,s}^{}c_{i,s}.E_{i,s}+ 
\sum_{l}^{}c_{l}.P_{l}+ 
 \right.
\end{equation*}
\begin{equation}
\left.\sum_{i,r,t}^{}w_{t}\left(\sum_{i,r}^{}o_{i,r}.g_{i,r,t}+\sum_{k}^{}o_{k}.f_{k,t}  \right)   \right]
\end{equation}

A set of constraints are also added to the optimization problem. One of the constraints is that demand is inelastic and must therefore be met completely at each time-step. Other constraints represent different physical and societal limitations such as the maximum renewable potential in every node, maximum transmission expansion, available renewable and non-renewable resources, maximum storage discharge and charge dispatch, and maximum carbon emissions. The objective function and all the constraints are linear, which leads to a linear programme (LP). 
The power flow in the network goes through two main elements: transmission network, and distribution grid. 
Transmission network is comprised of High Voltage Alternating Current (HVAC) lines connecting high-voltage (HV) buses together, as shown in the simplified example in Figure S1. Distribution grid is represented as a single bidirectional connections between each HV bus to its corresponding low-voltage (LV) bus in the same node. 
The simplest way to model power flow is to use the transport model, also known as network flow model. In this case we ignore all physical features of the power transmission such as line resistance and impedance, and only enforce ‘conservation of power’. This is done through the nodal power balance constraint that is modelled with Kirchhoff’s Current Law (KCL), as shown Eq.(2). The KCL constraint ensures that the total inflow power at each bus is equal to the total outflow power plus consumed power.
\begin{equation}
p_{i} = \sum_{l}^{} K_{il}.p_{l}   \quad\quad   \forall i \in \mathcal{N}
\end{equation}
where \(p_{i}\) is the active power injected or consumed at node \(i\), and \(K\) is the incidence matrix of the network graph which summarizes all connections between other nodes and node \(i\) as: not connected (0), connected with start at \(i\) (+1), and connected with end at \(i\) (-1). The power flowing through every line \(p_{l,t}\) is limited by the capacity of the line \(P_{l}\) as shown in Eq.(3), a capacity which is co-optimized if transmission expansion is allowed, as shown in Eq.(4).
\begin{equation}
\left| p_{l,t}  \right|\le \bar{p_{l}}P_{l} \quad\quad   \forall l,t
\end{equation}
The inclusion of  \(\bar{p_{l}}\) as an extra per-unit security margin on the line capacity serves to provide a buffer to account for potential failures of individual circuits (as per the N-1 criterion) and reactive power flows. 
\begin{equation}
\sum_{l}^{} l_{l}.P_{l}\le CAP_{LV} \quad\leftrightarrow \quad\mu_{LV}
\end{equation}
where the sum of transmission capacities \(P_{l}\) multiplied by the lengths \(l_{l}\) is bounded by a transmission volume cap of \(CAP_{LV}\). The Lagrange/KKT multiplier \(\mu_{LV}\) represents the shadow price of a marginal increase in transmission volume. As discussed in previous studies \citeS{neumann_2022assessments}, the transport model does not have enough constraints to produce a unique solution, and this results in arbitrary flows in the network because it does not cost the model anything to transmit power. It also means that the model does not capture possible bottlenecks in the network.

A better representation of the physical features of the power network, while keeping the problem linear, can be obtained with linearised optimal power flow (LOPF) equations, also known as DC power flow. This model adds linear constraints for Kirchhoff’s Voltage Law (KVL) to the KCL constraint of the transport model. The KVL constraint imposes the physical relation between the voltage differences at the extreme of a transmission line and the power flowing through it. The main assumptions behind LOPF are that power flows primarily accordingly to angle differences, no significant voltage shift occurs between the nodes, and line resistances are negligible compared to line reactance\citeS{neumann_2022assessments}. Along with some other simplifying assumptions, the KVL constraint can be linearised to calculate the power flow \(p_{i}\) in each line with Eq.(3):
\begin{equation}
p_{l} = \frac{\Theta_{i}-\Theta_{j}}{x_{l}}
\end{equation}
where \(\Theta_{i}\) and \(\Theta_{j}\) are voltage angles at nodes \(i\) and \(j\), and \(x_{l}\) is the line reactance. In the model, KVL is imposed by means of the cycle matrix where \(C_{lc}\) contains information on which line \(l\) is an element of a closed loop \(c\) \citeS{neumann_2022assessments}.
\begin{equation}
\sum_{l}^{} C_{lc}P_{l}x_{l}=0 \quad \forall c\in C
\end{equation}
Although LOPF improves upon the transport model by including reactance and voltage angles into the model, it also assumes negligible resistance, so power flow in the transmission network in our model is effectively lossless.
\begin{figure}[H]
\renewcommand*{\thefigure}{S\arabic{figure}}
\includegraphics[width=0.8\textwidth,center]{./figures/Supplementary_figures/figure_S0}
\caption{Schematic of connections between two high-voltage and low-voltage nodes in the network. Transmission lines are modeled using LOPF equations and line capacity \(P_{l}\) is optimized assuming zero resistive losses. Distribution grid connections are modeled using the Transport model and  capacity \(P_{l}\) is optimized assuming a constant efficiency of \(\eta_{D}\). }
\end{figure}
Modeling the network with LOPF is significantly more demanding computationally than using the transport model. To minimise computational complexity, only the transmission network, meaning all AC and DC HV transmission lines between HV buses, are modelled with LOPF. The assumptions behind LOPF are also more reasonable when looking at transmission lines. The distribution grid is modelled as a single connections between the HV bus and LV bus for each node with the transport model, so the only constraint that applies to it is the KCL constraint. To represent power flow losses in the distribution network, a constant efficiency is assumed for all distribution connections, which could be 100\% or 90\% (10\% power losses) as mentioned in the text. 

 
\subsection*{S2. Distribution grid modeling}

Some small-scale studies consider several levels for the grid with a high-high voltage level for transmission network and lower levels (high, medium, low) for distribution grid \citeS{muller2019}. In our model we consider only one HV level and one LV level per node. Another approach to modelling the distribution grid is to use 2 unidirectional connections instead of one bidirectional connection\citeS{clack2020}. This was tested for the scenario B under different transmission and distribution grid assumptions, as shown in Figure S2. The results showed no significant difference for system capacity mix. This is due to the fact that reverse flows in the distribution grid are very low and the system does not consider it cost-efficient to install any capacity for this direction. Hence, we follow a simple approach using one bidirectional connection.

\begin{figure}[H]
\renewcommand*{\thefigure}{S\arabic{figure}}
\includegraphics[width=1\textwidth,]{./figures/Supplementary_figures/figure_S1}
\caption{Comparison of installed capacity of major technologies for the power sector under different transmission grid allowance and distribution grid cost assumptions when the distribution grid is modeled as one bidirectional connection vs. as two unidirectional connections. }
\end{figure}

A number of studies including grid reinforcement costs are shown in Table S1. Some of these studies represent grid costs as \texteuro /m for distribution lines, or as \texteuro /kWh for the amount of energy transfer through the distribution grid. To convert these costs to the unit \texteuro /kW, the following is assumed for the European distribution grid : 300 million customers, 10 millions km of power lines, 2800 annual TWh demand \citeS{rullaud2020distribution} and 542 GW peak hourly demand \citeS{ENTSOE_2017}

Let us assume that we have a cost of 1 \texteuro /km for distribution grid. We can multiply this by the number 10 million (km), which would result in 10 million \texteuro  as the total cost of the distribution grids in Europe. We can then divide this number by the European energy system peak load, which is 542 GW, and the resulting number could be considered the cost of the distribution grid in \texteuro /kW. This methodology is rather crude, so only the studies where the numbers fall within an established criterion are presented here.

\begin{table}[H]
\renewcommand*{\thetable}{S\arabic{table}}
\centering
\caption{An overview of distribution grid reinforcement costs from selected studies}
\scriptsize
\begin{tabular}{p{3cm}p{3.2cm}p{1.5cm}p{1.5cm}p{1.5cm}p{1.5cm}}
\toprule
References & Capital cost                  (\texteuro /kW) & Lifetime (years) & Discount rate \\
\hline
Meunier et al. (2021) \citeS{meunier2021cost} & 15- 32 (from   per dwelling) & 33 & - \\
Miller et. All (2022) \citeS{miller2022grid} & 110 (from line costs) & - & - \\
Allard et al. (2020) \citeS{allard2020considering} & 148-232  (from line costs) & - & 8\% \\
Study by Imperial College, NERA and DNV GL (2014) \citeS{european2014integration} & 75-270  (from line costs) & - & 10\% \\
Gupta et al. (2021) \citeS{gupta_2021} & 51-220,1385,143 (reinforcement for PV, HP, EV) & 40-50 & 3.80\% \\
Lumbreras et. Al (2018) \citeS{lumbreras2018} & 0-80 (reinforcement for PV) & - & -  \\
\bottomrule
\end{tabular}
\end{table}

\subsection*{S3. Sensitivity of system capacity mix to spatial resolution}

\begin{figure}[H]
\renewcommand*{\thefigure}{S\arabic{figure}}
\includegraphics[width=0.8\textwidth,center]{./figures/Supplementary_figures/figure_S2}
\caption{Comparison of installed capacity of major technologies for the scenario B with distributed generation and storage under different spatial resolutions with 37, 181, 256, and 512 nodes}

\end{figure}

\subsection*{S4. Sensitivity of installed distributed solar capacity to grid costs and losses}

\begin{figure}[H]
\renewcommand*{\thefigure}{S\arabic{figure}}
\includegraphics[width=0.55\textwidth, center]{./figures/Supplementary_figures/figure_S3}
\caption{Sensitivity of the installed capacity of distributed solar for scenario B to distribution grid costs of 0, 500, and 2500 \texteuro /kW, and distribution power loss of 0, 10\%, 25\%. The difference for grid cost factor of 0.01 and 1 is negligible, so lower costs than base cost (500 \texteuro /kW) are ignored in the analysis.}

\end{figure}

\subsection*{S5. Sensitivity of system cost savings from distributed solar to emissions target}

\begin{figure}[H]
\renewcommand*{\thefigure}{S\arabic{figure}}
\includegraphics[width=0.8\textwidth, center]{./figures/Supplementary_figures/figure_S15}
\caption{Total costs of scenario B with and without distributed technologies for a) 95\% carbon emissions reduction compared to 1990 levels, and b) 80\% carbon emissions reduction compared to 1990 levels. The cost savings and system generation mix remain unchanged, showing that the results are robust for lower emission targets. }
\end{figure}

\subsection*{S6. Changes in installed capacity and costs of each technology for all scenarios}

\begin{figure}[H]
\renewcommand*{\thefigure}{S\arabic{figure}}
\includegraphics[width=0.9\textwidth,]{./figures/Supplementary_figures/figure_S4}
\caption{System capacity mix for scenarios A, B and C: with and without distributed generation.}
\end{figure}

\begin{figure}[H]
\renewcommand*{\thefigure}{S\arabic{figure}}
\includegraphics[width=0.7\textwidth, center]{./figures/Supplementary_figures/figure_S4_1}
\caption{Changes in system costs when including distributed solar and home batteries. Distributed solar is the main technology with increasing costs. Solar utility and distribution grid are the main components with decreasing costs.}
\end{figure}



\subsection*{S7. Energy generation mix for all scenarios for Italy, Germany, and the whole system}

\begin{figure}[H]
\renewcommand*{\thefigure}{S\arabic{figure}}
\includegraphics[width=0.8\textwidth, center]{./figures/Supplementary_figures/figure_S6_A1}
\includegraphics[width=0.8\textwidth,center]{./figures/Supplementary_figures/figure_S6_A2}
\includegraphics[width=0.8\textwidth,center]{./figures/Supplementary_figures/figure_S6_A3}
\caption{Energy generation mix for a week in Summer without(top) and with (bottom) distributed solar and batteries for scenario A}
\end{figure}

\begin{figure}[H]
\renewcommand*{\thefigure}{S\arabic{figure}}

\includegraphics[width=0.8\textwidth,center]
{./figures/Supplementary_figures/figure_S6_B1}
\includegraphics[width=0.8\textwidth,center]{./figures/Supplementary_figures/figure_S6_B2}
\includegraphics[width=0.8\textwidth,center]{./figures/Supplementary_figures/figure_S6_B3}
\caption{Energy generation mix for a week in Summer without(top) and with (bottom) distributed solar and batteries for scenario B}
\end{figure}

\begin{figure}[H]
\renewcommand*{\thefigure}{S\arabic{figure}}

\includegraphics[width=0.8\textwidth,center]{./figures/Supplementary_figures/figure_S6_C1}
\includegraphics[width=0.8\textwidth,center]{./figures/Supplementary_figures/figure_S6_C2}
\includegraphics[width=0.8\textwidth,center]{./figures/Supplementary_figures/figure_S6_C3}
\caption{Energy generation mix for a week in Summer without(top) and with (bottom) distributed solar and batteries for scenario C}
\end{figure}



\begin{figure}[H]
\renewcommand*{\thefigure}{S\arabic{figure}}

\includegraphics[width=0.55\textwidth,center]{./figures/Supplementary_figures/figure_S6_mapA}
\caption{Regional map for Scenario A with distributed generation showing share of major technologies in total annual electricity generation}
\end{figure}

\begin{figure}[H]
\renewcommand*{\thefigure}{S\arabic{figure}}

\includegraphics[width=0.55\textwidth,center]{./figures/Supplementary_figures/figure_S6_mapB}
\caption{Regional map for Scenario B with distributed generation showing share of major technologies in total annual electricity generation}
\end{figure}

\begin{figure}[H]
\renewcommand*{\thefigure}{S\arabic{figure}}

\includegraphics[width=0.55\textwidth,center]{./figures/Supplementary_figures/figure_S6_mapC}
\caption{Regional map for Scenario C with distributed generation showing share of major technologies in total annual electricity generation}
\end{figure}

\begin{figure}[H]
\renewcommand*{\thefigure}{S\arabic{figure}}

\includegraphics[width=0.55\textwidth,center]{./figures/Supplementary_figures/figure_S6_mapD}
\caption{Regional map for Scenario D with distributed generation showing share of major technologies in total annual electricity generation}
\end{figure}

\subsection*{S8. Installed capacity of different technologies to maximum installable capacity}

\begin{figure}[H]
\renewcommand*{\thefigure}{S\arabic{figure}}

\includegraphics[width=1\textwidth,]{./figures/Supplementary_figures/figure_S11}
\caption{Map of installed capacity vs. maximum installable capacity for distributed solar, utility solar, and wind in scenarios A, B, and C. All the capacities increase for the sector-coupled scenario C to meet additional demand. Installed capacity of distributed solar reaches maximum for most nodes in scenario C, and about half the nodes for scenario B}
\end{figure}

\subsection*{S9. Transmission network role}

\begin{figure}[H]
\renewcommand*{\thefigure}{S\arabic{figure}}

\includegraphics[width=1\textwidth,]{./figures/Supplementary_figures/figure_S14}   %S12
\caption{Sensitivity of the installed capacity and the energy generated/transmitted for solar utility, utility batteries, distributed solar, home batteries, and the distribution grid to transmission network expansion allowance for top) scenario A and bottom) scenario B. All figures are for the case when the scenario includes distributed generation and storage. Both scenario A and scenario B show a reduction in installed capacities of utility solar and utility battery when the transmission network allowance goes from 10\% of the current network capacity to 50\%, and to 'no-limit' expansion optimised by the model. Energy generation figures both show that wind generation is increasing for higher transmission network capacity, compensating for the reduction in solar energy. Due to easy transportation of energy from locations with high wind potential, although installed wind capacity decreases for optimal transmission allowances, wind energy generation continues to increase. The transmission network capacity is equal to 1.7 times the original transmission network for 'no-limit' expansion. }
\end{figure}

\begin{figure}[H]
\renewcommand*{\thefigure}{S\arabic{figure}}

\includegraphics[width=0.9\textwidth, center]{./figures/Supplementary_figures/figure_S13}   %S12
\caption{An overall look at the interdependence of transmission network and distribution network for scenario B for different transmission expansion volumes (10\%, 50\%, and no-limit expansion optimised by model). Transmission network's capacity and energy transmission do not show a noticeable change when distributed generation and storage technologies are not available. Distribution network's capacity and energy transmission shows a weak dependency on transmission network capacity, with both capacity and energy transmission increasing as the transmission network is allowed to expand more.}
\end{figure}

\begin{figure}[H]
\renewcommand*{\thefigure}{S\arabic{figure}}

\includegraphics[width=0.9\textwidth, center]{./figures/Supplementary_figures/figure_S9_3}   %S12
\caption{Changes in the installed capacity and total energy transmission through the transmission network for different distribution grid costs and losses. The transmission network is allowed to expand without limit in all cases. The figures show that 1) assumptions on distribution cost have no impact on transmission grid optimal capacity and total transmitted energy, and 2) assumption on distribution losses has a small impact as more energy needs to be transmitted to make up for distribution loss. 
}
\end{figure}

\subsection*{S10. Effect of distributed technologies on electricity transmission, dominant cycle, and price }

\begin{figure}[H]
\renewcommand*{\thefigure}{S\arabic{figure}}

\includegraphics[width=0.75\textwidth, center]{./figures/Supplementary_figures/figure_S7}
\caption{Duration curve for energy transfer in the distribution grid for entire system, Germany, and Spain. As expected, the LV demand peak reduction is highest for Spain, which has a higher installed capacity of distributed solar. }
\end{figure}

\begin{figure}[H]
\renewcommand*{\thefigure}{S\arabic{figure}}

\includegraphics[width=1\textwidth,]{./figures/Supplementary_figures/figure_S7_2}
\caption{Comparison of electricity price for scenario B with and without distributed technologies for different nodes. When distributed generation and storage are available, the electricity price has a lower variation. The low electricity prices in midday (known as duck curve) can be seen for all nodes and happens on both the LV buses and the HV buses.  }
\end{figure}

\begin{figure}[H]
\renewcommand*{\thefigure}{S\arabic{figure}}

\includegraphics[width=1\textwidth,]{./figures/Supplementary_figures/figure_S16}
\caption{Fourier power spectra of top) the distribution grid energy transfer time series from high-voltage (HV) to low-voltage (LV) and vice versa for scenario B and scenario C. The HV to LV graphs show three dominant cycles : (1) 6 hours, which is the average time solar power is available, (2) 12 hours, which shows the daily cycle of demand during day and night, and (3) 90 hours. The half-day, daily, and 3-day cycle can still be seen in the LV to HV (lower right) figure. Note: using a 1 year time series limits our ability to see the seasonal peak. Bottom figures show Fourier power spectra of time series for b) battery charger, battery discharger, and battery store hourly fill levels; c) pumped hydro storage (PHS) charger, PHS discharger, and PHS store hourly fill levels; d) battery electric vehicle (BEV) charger, vehicle to grid (V2G), and electric vehicle store hourly fill levels; and e) thermal storage charger, thermal storage discharger, and thermal store hourly fill levels.  
}
\end{figure}



\subsection*{S12. Role of distributed storage in the system}

\begin{figure}[H]
\renewcommand*{\thefigure}{S\arabic{figure}}

\includegraphics[width=0.8\textwidth,]{./figures/Supplementary_figures/figure_S8_0}
\caption{Sensitivity of installed capacity of solar utility, utility batteries, distributed solar, home batteries, the distribution grid, and onshore wind for scenario B to battery storage costs. The costs for battery inverter (160 \texteuro /KW for utility battery  \protect\citeS{dea_table} and 228 \texteuro /KW for home battery  \protect\citeS{dea_table,ram2018global}) and lithium ion battery (142 \texteuro /KW for utility battery  \protect\citeS{dea_table} and 203 \texteuro /KW for home battery  \protect\citeS{dea_table,ram2018global}) assumed here are consistent with most studies  \protect\citeS{mauler2021battery}. However, considering the importance of their role in increasing the cost-efficiency of solar PV, a sensitivity case with utility battery inverter cost of 100 \texteuro /KW  \protect\citeS{wraalsen2022multiple} and utility battery cost of 100 \texteuro /KW \protect\citeS{mauler2021battery} is shown here. Home battery costs, lowered with the same ratio as utility battery costs compared to default assumption, are 160 \texteuro /KW for inverter and 127 \texteuro /KW for battery. When battery costs are lower, there is respectively a 8.8\% and 6.3\% increase in utility solar installed capacity and distributed solar installed capacity. Cost reduction as a result of including distributed solar in the system goes from 1.42\% (scenario B) to 1.84\% with cheaper battery storage. }
\end{figure}

\begin{figure}[H]
\renewcommand*{\thefigure}{S\arabic{figure}}

\includegraphics[width=0.83\textwidth,]{./figures/Supplementary_figures/figure_S8_1}
\caption{Installed capacity of solar utility, utility batteries, distributed solar, home batteries, and the distribution grid for scenario B with/without home batteries and distributed solar. The figure shows that the system installs home batteries even without the presence of distributed solar.}
\end{figure}

\begin{figure}[H]
\renewcommand*{\thefigure}{S\arabic{figure}}

\includegraphics[width=0.9\textwidth, ]{./figures/Supplementary_figures/figure_S8_2}
\caption{Installed capacity of solar utility, utility batteries, distributed solar, home batteries, and the distribution grid for scenario C with and without the presence of different distributed storage technologies. There is no major difference observed without the presence of home batteries, and a small reduction when electric vehicle (EV) batteries and distributed thermal storage are also not included in the system. }
\end{figure}

\begin{figure}[H]
\renewcommand*{\thefigure}{S\arabic{figure}}

\includegraphics[width=0.9\textwidth, ]{./figures/Supplementary_figures/figure_S8_3}
\caption{Installed capacity of solar utility, utility batteries, distributed solar, home batteries, and the distribution grid for scenario D with and without the presence of different distributed storage technologies. There is a 80\% reduction in installed capacity of distributed solar when EV batteries are not included. Overall, There is a significant reduction of about 85\% in installed capacity of distributed solar when both EV batteries and thermal storage are not included in the system. This shows that the balancing provided by distributed storage majorly boosts the profitability of distributed solar in the system.}
\end{figure}

\begin{figure}[H]
\renewcommand*{\thefigure}{S\arabic{figure}}

\includegraphics[width=0.9\textwidth, center]{./figures/Supplementary_figures/figure_S8_4}
\caption{Changes in the operation of electric vehicles battery's (BEV) charging and vehicle-to-grid (V2G) energy transfer for scenario C with and without distributed solar assuming default or high distributed solar potential. There is a clear rise in the energy transfer for both technologies in summer as they help balance distributed PV, especially when more solar is available. The high installed capacity of distributed PV for scenario D with high distributed solar potential means that the usage is increased for both BEV charging and V2G to help balance the large PV production at the low-voltage level of the grid. }
\end{figure}


\subsection*{S13. Sensitivity analysis for sector-coupled scenario}

\begin{figure}[H]
\renewcommand*{\thefigure}{S\arabic{figure}}
\includegraphics[width=0.8\textwidth,center]{./figures/Supplementary_figures/figure_S13_0}
\caption{Sensitivity of installed distributed solar to distribution grid cost and distribution grid losses when default distributed PV potential (504 GW) is assumed. There is only a 8.6\% reduction in the installed capacity of distributed solar. This shows that 500 GW of distributed solar is still cost-efficient for the system with lower grid costs and no grid losses.  }
\end{figure}

\begin{figure}[H]
\renewcommand*{\thefigure}{S\arabic{figure}}
\includegraphics[width=0.8\textwidth,center]{./figures/Supplementary_figures/figure_S13_1}
\caption{Changes in heat generation mix for scenario C when distribution grid cost is 500 \texteuro /kW and power losses are 0\%. As mentioned in the paper, solar thermal is not selected in this case due to the fact that combined heat and power plants are more cost-efficient. }
\end{figure}

\begin{figure}[H]
\renewcommand*{\thefigure}{S\arabic{figure}}
\includegraphics[width=0.7\textwidth, center]{./figures/Supplementary_figures/figure_S13_2}
\caption{Comparison of system costs for sector-coupled scenario with high distributed PV potential (3000 GW) with and without distributed technologies where distribution grid losses are 7\% and distribution grid cost is 500 \texteuro /KW. Cost savings (2.1\%) are less significant compared to scenario D (3.7\% savings) where assumptions for the distribution grid are 10\% losses and 1000 \texteuro /KW costs. However, the system still installs 1900 GW of distributed PV, which is 43\% of the total installed solar capacity equal to 4400 GW. This shows that even with more conservative assumptions regarding distribution grid, distributed solar is still profitable for the system. }
\end{figure}



\subsection*{S14. Cost assumptions}
Costs for all technologies and the source for each data is available at Github repository of PyPSA Technology Data \citeS{pypsacosts}.

\fontsize{7}{0.1}\selectfont
\renewcommand*{\thetable}{S\arabic{table}}
\setcounter{table}{2}
\begin{longtblr}[
 caption = {Projected cost assumptions for major technologies in 2030.},
  label = none,
  entry = none,
]{
  rowhead=1,
  width = \linewidth,
  colspec = {Q[304]Q[240]Q[102]Q[292]},
  cell{2}{1} = {r=3}{},
  cell{5}{1} = {r=3}{},
  cell{8}{1} = {r=3}{},
  cell{11}{1} = {r=3}{},
  cell{14}{1} = {r=5}{},
  cell{19}{1} = {r=6}{},
  cell{25}{1} = {r=4}{},
  cell{29}{1} = {r=4}{},
  cell{33}{1} = {r=2}{},
  cell{35}{1} = {r=5}{},
  cell{40}{1} = {r=8}{},
  cell{48}{1} = {r=5}{},
  cell{53}{1} = {r=5}{},
  cell{58}{1} = {r=8}{},
  cell{66}{1} = {r=6}{},
  cell{72}{1} = {r=5}{},
  cell{77}{1} = {r=5}{},
  cell{82}{1} = {r=5}{},
  cell{87}{1} = {r=3}{},
  cell{90}{1} = {r=9}{},
  cell{99}{1} = {r=8}{},
  cell{107}{1} = {r=6}{},
  cell{113}{1} = {r=7}{},
  cell{120}{1} = {r=4}{},
  cell{124}{1} = {r=5}{},
  cell{129}{1} = {r=5}{},
  cell{134}{1} = {r=2}{},
  cell{136}{1} = {r=5}{},
  cell{141}{1} = {r=4}{},
  cell{145}{1} = {r=5}{},
  cell{150}{1} = {r=4}{},
  cell{154}{1} = {r=4}{},
  cell{158}{1} = {r=8}{},
  cell{166}{1} = {r=6}{},
  cell{172}{1} = {r=5}{},
  cell{177}{1} = {r=5}{},
  cell{182}{1} = {r=2}{},
  cell{184}{1} = {r=4}{},
  cell{188}{1} = {r=2}{},
  cell{190}{1} = {r=4}{},
  cell{194}{1} = {r=6}{},
  cell{200}{1} = {r=4}{},
  cell{204}{1} = {r=4}{},
  cell{208}{1} = {r=7}{},
  cell{215}{1} = {r=4}{},
  cell{219}{1} = {r=4}{},
  cell{223}{1} = {r=3}{},
  hline{1} = {-}{0.08em},
  hline{2,5,8,11,14,19,25,29,33,35,40,48,53,58,66,72,77,82,87,90,99,107,113,120,124,129,134,136,141,145,150,154,158,166,172,177,182,184,188,190,194,200,204,208,215,219,223,226-228} = {-}{},
}
Technology                         & Parameter                     & Value     & Unit                              \\
HVAC overhead                      & FOM                           & 2         & \%/year                           \\
                                   & investment                    & 432.97    & EUR/MW/km                         \\
                                   & lifetime                      & 40        & years                             \\
HVDC inverter pair                 & FOM                           & 2         & \%/year                           \\
                                   & investment                    & 162364.82 & EUR/MW                            \\
                                   & lifetime                      & 40        & years                             \\
HVDC overhead                      & FOM                           & 2         & \%/year                           \\
                                   & investment                    & 432.97    & EUR/MW/km                         \\
                                   & lifetime                      & 40        & years                             \\
HVDC submarine                     & FOM                           & 0.35      & \%/year                           \\
                                   & investment                    & 471.16    & EUR/MW/km                         \\
                                   & lifetime                      & 40        & years                             \\
OCGT                               & FOM                           & 1.78      & \%/year                           \\
                                   & VOM                           & 4.5       & EUR/MWh                           \\
                                   & efficiency                    & 0.41      & per unit                          \\
                                   & investment                    & 435.24    & EUR/kW                            \\
                                   & lifetime                      & 25        & years                             \\
CCGT                               & VOM                           & 4.2       & EUR/MWh                           \\
                                   & c\_b                          & 2         & 50\degree C/100\degree C                        \\
                                   & c\_v                          & 0.15      & 50\degree C/100\degree C                        \\
                                   & efficiency                    & 0.58      & per unit                          \\
                                   & investment                    & 830       & EUR/kW                            \\
                                   & lifetime                      & 25        & years                             \\
PHS                                & FOM                           & 1         & \%/year                           \\
                                   & efficiency                    & 0.75      & per unit                          \\
                                   & investment                    & 2208.16   & EUR/kWel                          \\
                                   & lifetime                      & 80        & years                             \\
battery inverter                   & FOM                           & 0.34      & \%/year                           \\
                                   & efficiency                    & 0.96      & per unit                          \\
                                   & investment                    & 160       & EUR/kW                            \\
                                   & lifetime                      & 10        & years                             \\
battery storage                    & investment                    & 142       & EUR/kWh                           \\
                                   & lifetime                      & 25        & years                             \\
biomass                            & FOM                           & 4.53      & \%/year                           \\
                                   & efficiency                    & 0.47      & per unit                          \\
                                   & fuel                          & 7         & EUR/MWhth                         \\
                                   & investment                    & 2209      & EUR/kWel                          \\
                                   & lifetime                      & 30        & years                             \\
biomass CHP                        & FOM                           & 3.58      & \%/year                           \\
                                   & VOM                           & 2.1       & EUR/MWh\_e                        \\
                                   & c\_b                          & 0.46      & 40\degree C/80\degree C                       \\
                                   & c\_v                          & 1         & 40\degree C/80\degree C                       \\
                                   & efficiency                    & 0.3       & per unit                          \\
                                   & efficiency-heat               & 0.71      & per unit                          \\
                                   & investment                    & 3210.28   & EUR/kW\_e                         \\
                                   & lifetime                      & 25        & years                             \\
biomass boiler                     & FOM                           & 7.49      & \%/year                           \\
                                   & efficiency                    & 0.86      & per unit                          \\
                                   & investment                    & 649.3     & EUR/kW\_th                        \\
                                   & lifetime                      & 20        & years                             \\
                                   & pelletizing cost              & 9         & EUR/MWh\_pellets                  \\
central air-sourced heat pump      & FOM                           & 0.23      & \%/year                           \\
                                   & VOM                           & 2.51      & EUR/MWh\_th                       \\
                                   & efficiency                    & 3.6       & per unit                          \\
                                   & investment                    & 856.25    & EUR/kW\_th                        \\
                                   & lifetime                      & 25        & years                             \\
central gas CHP                    & FOM                           & 3.32      & \%/year                           \\
                                   & VOM                           & 4.2       & EUR/MWh                           \\
                                   & c\_b                          & 1         & 50\degree C/100\degree C                        \\
                                   & c\_v                          & 0.17      & per unit                          \\
                                   & efficiency                    & 0.41      & per unit                          \\
                                   & investment                    & 560       & EUR/kW                            \\
                                   & lifetime                      & 25        & years                             \\
                                   & p\_nom\_ratio                 & 1         & per unit                          \\
central gas CHP CC                 & FOM                           & 3.32      & \%/year                           \\
                                   & VOM                           & 4.2       & EUR/MWh                           \\
                                   & c\_b                          & 1         & 50\degree C/100\degree C                        \\
                                   & efficiency                    & 0.41      & per unit                          \\
                                   & investment                    & 560       & EUR/kW                            \\
                                   & lifetime                      & 25        & years                             \\
central gas boiler                 & FOM                           & 3.8       & \%/year                           \\
                                   & VOM                           & 1         & EUR/MWh\_th                       \\
                                   & efficiency                    & 1.04      & per unit                          \\
                                   & investment                    & 50        & EUR/kW\_th                        \\
                                   & lifetime                      & 25        & years                             \\
central ground-sourced heat pump   & FOM                           & 0.39      & \%/year                           \\
                                   & VOM                           & 1.25      & EUR/MWh\_th                       \\
                                   & efficiency                    & 1.73      & per unit                          \\
                                   & investment                    & 507.6     & EUR/kW\_th excluding drive energy \\
                                   & lifetime                      & 25        & years                             \\
central resistive heater           & FOM                           & 1.7       & \%/year                           \\
                                   & VOM                           & 1         & EUR/MWh\_th                       \\
                                   & efficiency                    & 0.99      & per unit                          \\
                                   & investment                    & 60        & EUR/kW\_th                        \\
                                   & lifetime                      & 20        & years                             \\
central solar thermal              & FOM                           & 1.4       & \%/year                           \\
                                   & investment                    & 140000    & EUR/1000m2                        \\
                                   & lifetime                      & 20        & years                             \\
central solid biomass CHP          & FOM                           & 2.87      & \%/year                           \\
                                   & VOM                           & 4.58      & EUR/MWh\_e                        \\
                                   & c\_b                          & 0.35      & 50\degree C/100\degree C                      \\
                                   & c\_v                          & 1         & 50\degree C/100\degree C                      \\
                                   & efficiency                    & 0.27      & per unit                          \\
                                   & efficiency-heat               & 0.82      & per unit                          \\
                                   & investment                    & 3349.49   & EUR/kW\_e                         \\
                                   & lifetime                      & 25        & years                             \\
                                   & p\_nom\_ratio                 & 1         & per unit                          \\
central solid biomass CHP CC       & FOM                           & 2.87      & \%/year                           \\
                                   & VOM                           & 4.58      & EUR/MWh\_e                        \\
                                   & c\_b                          & 0.35      & 50\degree C/100\degree C                      \\
                                   & c\_v                          & 1         & 50\degree C/100\degree C                      \\
                                   & efficiency                    & 0.27      & per unit                          \\
                                   & efficiency-heat               & 0.82      & per unit                          \\
                                   & investment                    & 4921.02   & EUR/kW\_e                         \\
                                   & lifetime                      & 25        & years                             \\
central water tank storage         & FOM                           & 0.55      & \%/year                           \\
                                   & investment                    & 0.54      & EUR/kWhCapacity                   \\
                                   & lifetime                      & 25        & years                             \\
                                   & FOM                           & 2         & \%/year                           \\
                                   & investment                    & 67.63     & EUR/m\^{}3-H2O                    \\
                                   & lifetime                      & 30        & years                             \\
coal                               & CO2 intensity                 & 0.34      & tCO2/MWh\_th                      \\
                                   & FOM                           & 1.6       & \%/year                           \\
                                   & VOM                           & 3.5       & EUR/MWh\_e                        \\
                                   & efficiency                    & 0.33      & per unit                          \\
                                   & fuel                          & 8.15      & EUR/MWh\_th                       \\
                                   & investment                    & 3845.51   & EUR/kW\_e                         \\
                                   & lifetime                      & 40        & years                             \\
decentral CHP                      & FOM                           & 3         & \%/year                           \\
                                   & discount rate                 & 0.04      & per unit                          \\
                                   & investment                    & 1400      & EUR/kWel                          \\
                                   & lifetime                      & 25        & years                             \\
decentral air-sourced heat pump    & FOM                           & 3         & \%/year                           \\
                                   & discount rate                 & 0.04      & per unit                          \\
                                   & efficiency                    & 3.6       & per unit                          \\
                                   & investment                    & 850       & EUR/kW\_th                        \\
                                   & lifetime                      & 18        & years                             \\
decentral gas boiler               & FOM                           & 6.69      & \%/year                           \\
                                   & discount rate                 & 0.04      & per unit                          \\
                                   & efficiency                    & 0.98      & per unit                          \\
                                   & investment                    & 296.82    & EUR/kW\_th                        \\
                                   & lifetime                      & 20        & years                             \\
decentral gas boiler connection    & investment                    & 185.51    & EUR/kW\_th                        \\
                                   & lifetime                      & 50        & years                             \\
decentral ground-sourced heat pump & FOM                           & 1.82      & \%/year                           \\
                                   & discount rate                 & 0.04      & per unit                          \\
                                   & efficiency                    & 3.9       & per unit                          \\
                                   & investment                    & 1400      & EUR/kW\_th                        \\
                                   & lifetime                      & 20        & years                             \\
decentral oil boiler               & FOM                           & 2         & \%/year                           \\
                                   & efficiency                    & 0.9       & per unit                          \\
                                   & investment                    & 156.01    & EUR/kWth                          \\
                                   & lifetime                      & 20        & years                             \\
decentral resistive heater         & FOM                           & 2         & \%/year                           \\
                                   & discount rate                 & 0.04      & per unit                          \\
                                   & efficiency                    & 0.9       & per unit                          \\
                                   & investment                    & 100       & EUR/kWhth                         \\
                                   & lifetime                      & 20        & years                             \\
decentral solar thermal            & FOM                           & 1.3       & \%/year                           \\
                                   & discount rate                 & 0.04      & per unit                          \\
                                   & investment                    & 270000    & EUR/1000m2                        \\
                                   & lifetime                      & 20        & years                             \\
decentral water tank storage       & FOM                           & 1         & \%/year                           \\
                                   & discount rate                 & 0.04      & per unit                          \\
                                   & investment                    & 18.38     & EUR/kWh                           \\
                                   & lifetime                      & 20        & years                             \\
direct air capture                 & FOM                           & 4.95      & \%/year                           \\
                                   & compression-electricity-input & 0.15      & MWh/tCO2                          \\
                                   & compression-heat-output       & 0.2       & MWh/tCO2                          \\
                                   & electricity-input             & 0.4       & MWh\_el/t\_CO2                    \\
                                   & heat-input                    & 1.6       & MWh\_th/t\_CO2                    \\
                                   & heat-output                   & 1         & MWh/tCO2                          \\
                                   & investment                    & 6000000   & EUR/(tCO2/h)                      \\
                                   & lifetime                      & 20        & years                             \\
electricity distribution grid      & FOM                           & 2         & \%/year                           \\
                                   & investment                    & 500       & EUR/kW                            \\
                                   & lifetime                      & 40        & years                             \\
                                   & FOM                           & 2         & \%/year                           \\
                                   & investment                    & 140       & EUR/kW                            \\
                                   & lifetime                      & 40        & years                             \\
electrolysis                       & FOM                           & 2         & \%/year                           \\
                                   & efficiency                    & 0.8       & per unit                          \\
                                   & efficiency-heat               & 0.17      & per unit                          \\
                                   & investment                    & 407.58    & EUR/kW\_e                         \\
                                   & lifetime                      & 30        & years                             \\
fuel cell                          & FOM                           & 5         & \%/year                           \\
                                   & c\_b                          & 1.25      & 50\degree C/100\degree C                        \\
                                   & efficiency                    & 0.58      & per unit                          \\
                                   & investment                    & 1100      & EUR/kW\_e                         \\
                                   & lifetime                      & 10        & years                             \\
gas                                & CO2 intensity                 & 0.2       & tCO2/MWh\_th                      \\
                                   & fuel                          & 20.1      & EUR/MWh\_th                       \\
home battery inverter              & FOM                           & 0.34      & \%/year                           \\
                                   & efficiency                    & 0.96      & per unit                          \\
                                   & investment                    & 228.06    & EUR/kW                            \\
                                   & lifetime                      & 10        & years                             \\
home battery storage               & investment                    & 202.9     & EUR/kWh                           \\
                                   & lifetime                      & 25        & years                             \\
hydro                              & FOM                           & 1         & \%/year                           \\
                                   & efficiency                    & 0.9       & per unit                          \\
                                   & investment                    & 2208.16   & EUR/kWel                          \\
                                   & lifetime                      & 80        & years                             \\
nuclear                            & FOM                           & 1.4       & \%/year                           \\
                                   & VOM                           & 3.5       & EUR/MWh\_e                        \\
                                   & efficiency                    & 0.33      & per unit                          \\
                                   & fuel                          & 2.6       & EUR/MWh\_th                       \\
                                   & investment                    & 7940.45   & EUR/kW\_e                         \\
                                   & lifetime                      & 40        & years                             \\
onshore wind                       & FOM                           & 1.22      & \%/year                           \\
                                   & VOM                           & 1.35      & EUR/MWh                           \\
                                   & investment                    & 1035.56   & EUR/kW                            \\
                                   & lifetime                      & 30        & years                             \\
offshore wind                      & FOM                           & 2.32      & \%/year                           \\
                                   & VOM                           & 0.02      & EUR/MWhel                         \\
                                   & investment                    & 1523.55   & EUR/kW\_e, 2020                   \\
                                   & lifetime                      & 30        & years                             \\
oil                                & CO2 intensity                 & 0.26      & tCO2/MWh\_th                      \\
                                   & FOM                           & 2.46      & \%/year                           \\
                                   & VOM                           & 6         & EUR/MWh                           \\
                                   & efficiency                    & 0.35      & per unit                          \\
                                   & fuel                          & 50        & EUR/MWhth                         \\
                                   & investment                    & 343       & EUR/kW                            \\
                                   & lifetime                      & 25        & years                             \\
ror                                & FOM                           & 2         & \%/year                           \\
                                   & efficiency                    & 0.9       & per unit                          \\
                                   & investment                    & 3312.24   & EUR/kWel                          \\
                                   & lifetime                      & 80        & years                             \\
solar-rooftop                      & FOM                           & 1.42      & \%/year                           \\
                                   & discount rate                 & 0.04      & per unit                          \\
                                   & investment                    & 636.66    & EUR/kW\_e                         \\
                                   & lifetime                      & 40        & years                             \\
solar-utility                      & FOM                           & 2.48      & \%/year                           \\
                                   & investment                    & 347.56    & EUR/kW\_e                         \\
                                   & lifetime                      & 40        & years                             \\
water tank charger                 & efficiency                    & 0.84      & per unit                          \\
water tank discharger              & efficiency                    & 0.84      & per unit                          
\end{longtblr}


\clearpage


